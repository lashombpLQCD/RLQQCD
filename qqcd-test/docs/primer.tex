\documentclass[12pt]{article}
\evensidemargin -1cm
\oddsidemargin -1cm
\textwidth 18cm
\begin{document}
\begin{center}
{\bf A primer for lattice QCD simulations}
\end{center}

\section{Quenched gauge field configurations}

The gauge action is ($\xi_4=1$, but showing it makes the expression look nicer)
\[
S_G[U] = \sum_{x,\mu,\nu>\mu}\frac{\beta}{u_\mu^2u_\nu^2}
         \frac{\xi_1\xi_2\xi_3\xi_4}{\xi_\mu^2\xi_\nu^2}
         \left[P_{\mu\nu}(x)
         +\frac{g_\mu}{3}\left(P_{\mu\nu}(x)
         -\frac{R_{\mu\nu}(x)}{4u_\mu^2}\right)
         +\frac{g_\nu}{3}\left(P_{\mu\nu}(x)
         -\frac{R_{\nu\mu}(x)}{4u_\nu^2}\right)
\right]
\]
with
\[
P_{\mu\nu}(x) = 1-\frac{1}{3}{\rm ReTr}\left[U_\mu(x)U_\nu(x+\mu)
                U_\mu^\dagger(x+\nu)U_\nu^\dagger(x)\right]
\]
is the $\mu\times\nu = 1\times1$ plaquette and
\[
R_{\mu\nu}(x) = 1-\frac{1}{3}{\rm ReTr}\left[U_\mu(x)U_\mu(x+\mu)U_\nu(x+2\mu)
                U_\mu^\dagger(x+\mu+\nu)U_\mu^\dagger(x+\nu)U_\nu^\dagger(x)
                \right]
\]
is the $\mu\times\nu = 2\times1$ rectangle (not $1\times2$ rectangle).

The user-defined parameters are:
\begin{eqnarray}
\beta &=& {\rm bare~gauge~field~coupling} \nonumber \\
g_\mu &=& \left\{\begin{array}{ll}
 0, & {\rm for~no~improvement~in~the~}\mu{\rm ~direction} \\
 1, & {\rm for~improvement~in~the~}\mu{\rm ~direction}
 \end{array}\right. \nonumber \\
\xi_\mu &=& {\rm lattice~spacings~in~units~of~temporal~spacing}
            \equiv a_\mu/a_t \nonumber \\
u_\mu &=& {\rm tadpole~factor~in~}\mu{\rm ~direction} \nonumber
\end{eqnarray}

\section{Fermion propagation}

The Wilson+clover fermion action is
\begin{eqnarray}
S_F[\bar\psi,\psi,U] &=& \frac{1}{2\kappa}\sum_{x,y}\bar\psi(x)
                         \left[A(x,y)-\kappa B(x,y)\right]\psi(y) \nonumber \\
A(x,y) &=& \delta_{x,y}\left[1+\frac{\kappa c_{SW}}{2}\sum_{\mu,\nu}
           \frac{r}{\xi_\mu\xi_\nu}i\sigma_{\mu\nu}F_{\mu\nu}(x)\right]
           \nonumber \\
B(x,y) &=& \sum_\mu\frac{1}{\xi_\mu^2u_\mu}\left[(r-\xi_\mu\gamma_\mu)
           U_\mu(x)\delta_{x+\mu,y}
           +(r+\xi_\mu\gamma_\mu)U_\mu^\dagger(y)\delta_{x-\mu,y}\right]
           \nonumber \\
F_{\mu\nu}(x) &=& \frac{1}{8u_\mu^2u_\nu^2}[Q_{\mu\nu}(x)-Q_{\mu\nu}^\dagger
           (x)] \nonumber\\
Q_{\mu\nu}(x) &=&
  U_\mu(x)U_\nu(x+\mu)U_\mu^\dagger(x+\nu)U_\nu^\dagger(x) \nonumber \\
&&+U_\nu(x)U_\mu^\dagger(x-\mu+\nu)U_\nu^\dagger(x-\mu)U_\mu(x-\mu) \nonumber \\
&&+U_\mu^\dagger(x-\mu)U_\nu^\dagger(x-\mu-\nu)U_\mu(x-\mu-\nu)U_\nu(x-\nu)
  \nonumber \\
&&+U_\nu^\dagger(x-\nu)U_\mu(x-\nu)U_\nu(x+\mu-\nu)U_\mu^\dagger(x) \nonumber
\end{eqnarray}
where
\[
\sigma_{\mu\nu} = \frac{i}{2}[\gamma_\mu,\gamma_\nu] = -\sigma_{\nu\mu}
\]
The definition agrees with the isotropic result of
Luscher, Sint, Sommer and Weisz, hep-lat/9605038,
who carefully record their conventions for Dirac matrices and
$\sigma_{\mu\nu}$.

For the anisotropic action, Alford, Klassen and Lepage NPB496, 377 (1997)
used $r=1$ but Groote and Shigemitsu PRD62, 014508 (2000) used
$r=a_s/a_t$ (for a spatially isotropic action).  See Aoki et al hep-lat/0107009
for a discussion of both options; they conclude in favour of Groote and
Shigemitsu.  However, Harada, Kronfeld, Matsufuru, Nakajima and Onogi,
hep-lat/0103026 seem to make the opposite choice.
It is not immediately obvious how to generalize Groote and Shigemitsu, to
a spatially-anisotropic lattice.

In my codes, the convention for Dirac algebra is
\[
\gamma_1 = \left(\begin{array}{cccc} 0 & 0 & 0 & 1 \\ 
           0 & 0 & 1 & 0 \\ 0 & 1 & 0 & 0 \\ 1 & 0 & 0 & 0 \end{array}\right),
\gamma_2 = \left(\begin{array}{cccc} 0 & 0 & 0 & -i \\ 
           0 & 0 & i & 0 \\ 0 & -i & 0 & 0 \\ i & 0 & 0 & 0 \end{array}\right),
\gamma_3 = \left(\begin{array}{cccc} 0 & 0 & 1 & 0 \\ 
           0 & 0 & 0 & -1 \\ 1 & 0 & 0 & 0 \\ 0 & -1 & 0 & 0 \end{array}\right),
\gamma_4 = \left(\begin{array}{cccc} 1 & 0 & 0 & 0 \\ 
           0 & 1 & 0 & 0 \\ 0 & 0 & -1 & 0 \\ 0 & 0 & 0 & -1 \end{array}\right).
\]
which leads to
\[
\gamma_5 = \gamma_1\gamma_2\gamma_3\gamma_4
         = \left(\begin{array}{cccc} 0 & 0 & -i & 0 \\ 
           0 & 0 & 0 & -i \\ i & 0 & 0 & 0 \\ 0 & i & 0 & 0 \end{array}\right)
\]
A useful comparison for the isotropic version of this action is
Luscher, Sint, Sommer and Weisz, hep-lat/9605038.

\end{document}
